\documentclass{article}
\usepackage[utf8]{inputenc}
\usepackage{natbib}
\usepackage{amsmath}
\usepackage{amssymb}

\title{\textbf{MATH 307: Group Homework 6}}
\author{John Mays, Henry Zhong, Zhitao Chen, Huajin Xin}
\date{03/19/21, Dr. Guo}

\begin{document}

\maketitle

\section*{Problem 1}
\subsubsection*{Reflection: }
For a reflection across the y-axis, the x-component of a vector must become -x\\
$A\begin{pmatrix}x\\ y\end{pmatrix} = \begin{pmatrix}-x\\ y\end{pmatrix}$\\
If $A = \begin{pmatrix}-1 & 0\\ 0 & 1\end{pmatrix}$, then $A\begin{pmatrix}x\\ y\end{pmatrix}=\begin{pmatrix}-1(x)+0(x)\\ 0(y)+1(y)\end{pmatrix}=\begin{pmatrix}-x\\ y\end{pmatrix}$\\
Therefore $A$ is the reflection matrix.
\subsubsection*{Rotation: }
To obtain the rotation matrix, we can compute the default Cartesian rotation matrix with $\theta = \frac{\pi}{4}$\\
$B=\begin{pmatrix}\cos{\frac{\pi}{4}} & -\sin{\frac{\pi}{4}}\\ \sin{\frac{\pi}{4}} & \cos{\frac{\pi}{4}}\ \end{pmatrix}=\begin{pmatrix}\frac{1}{\sqrt{2}} & -\frac{1}{\sqrt{2}}\\ \frac{1}{\sqrt{2}} & \frac{1}{\sqrt{2}} \end{pmatrix}$
Therefore $B$ is the rotation matrix.
\subsubsection*{Entire Operation:}
The matrix for the entire operation is the reflection matrix times the rotation matrix,\\
$BA=\begin{pmatrix}\frac{1}{\sqrt{2}} & -\frac{1}{\sqrt{2}}\\ \frac{1}{\sqrt{2}} & \frac{1}{\sqrt{2}} \end{pmatrix}\begin{pmatrix}-1 & 0\\ 0 & 1\end{pmatrix}=
\begin{pmatrix}(\frac{1}{\sqrt{2}})(-1)+(-\frac{1}{\sqrt{2}})(0) & (\frac{1}{\sqrt{2}})(0)+(-\frac{1}{\sqrt{2}})(1)\\ (\frac{1}{\sqrt{2}})(-1)+(\frac{1}{\sqrt{2}})(0) & (\frac{1}{\sqrt{2}})(0)+(\frac{1}{\sqrt{2}})(1) \end{pmatrix}=
\begin{pmatrix} -\frac{\sqrt{2}}{2} & -\frac{\sqrt{2}}{2}\\ -\frac{\sqrt{2}}{2} & \frac{\sqrt{2}}{2}
\end{pmatrix}
$
\pagebreak

\section*{Problem 2}
Equation in question: $(A+B)^2=A^2+2AB+B^2$ for two square matrices of the same size, $A$ and $B$.
\subsubsection*{LHS:}
The $ij$-th entry of $(A+B) $ is $a_{ij}+b_{ij}$\\
%The $ij$-th of $(A+B)^2$ can be thought of with the matrix-multiplication product of $(A+B)(A+B)$:\\
The $ij$-th entry of $(A+B)^2 = (A+B)(A+B)$ is $\sum_{k=1}^n (a_{ik}+b_{ik})(a_{kj}+b_{kj})= \sum_{k=1}^n a_{ik}a_{kj}+a_{ik}b_{kj}+b_{ik}a_{kj}+b_{ik}b_{kj}$
\subsubsection*{RHS:}
The $ij$-th entry of $A^2$ is $\sum_{k=1}^n a_{ik}a_{kj}$\\
It follows that the $ij$-th entry of $B^2$ is $\sum_{k=1}^n b_{ik}b_{kj}$\\
The $ij$-th entry of $AB$ is $\sum_{k=1}^n a_{ik}b_{kj}$\\
Therefore the $ij$-th entry of $A^2+2AB+B^2$ is $\sum_{k=1}^n a_{ik}a_{kj} + 2a_{ik}b_{kj}+ b_{ik}b_{kj}$
\subsubsection*{Conclusion:}
The respective $ij$-th entry of the LHS and the RHS of the equation are not equivalent, therefore $(A+B)^2=A^2+2AB+B^2$ is not true for two square matrices, $A$ and $B,$ of the same size.

\end{document}