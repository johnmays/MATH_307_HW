\documentclass[fleqn]{article}
\usepackage[utf8]{inputenc}
\usepackage{natbib}
\usepackage{amsmath}
\usepackage{amssymb}

\title{\textbf{MATH 307: Individual Homework 12}}
\author{John Mays}
\date{03/22/21, Dr. Guo}

\begin{document}

\maketitle
%\begin{pmatrix}\end{pmatrix}

\section*{Problem 1}
$A = \begin{pmatrix}1 & i \\ 1 & -i\end{pmatrix}$\\
Assume $B$ is a right inverse of $A \implies AB=I=\begin{pmatrix}1 & 0 \\ 0 & 1\end{pmatrix}. $\\ 
Then,\\
$\begin{pmatrix}1 & i\end{pmatrix}b_1=1$\\
$\begin{pmatrix}1 & i\end{pmatrix}b_2=0$\\
$\begin{pmatrix}1 & -i\end{pmatrix}b_1=0$\\
$\begin{pmatrix}1 & -i\end{pmatrix}b_2=1$\\
Therefore $b_1=\begin{pmatrix}\frac{1}{2}\\ -\frac{1}{2}i\end{pmatrix}$ and $b_2=\begin{pmatrix}\frac{1}{2}\\ \frac{1}{2}i\end{pmatrix}.$  So $B = \begin{pmatrix}\frac{1}{2} & \frac{1}{2}\\ -\frac{1}{2}i & \frac{1}{2}i\end{pmatrix}$.\\
\linebreak
To confirm that $B$ is a left inverse of $A,$ $BA$ should be equal to $I.$\\
\linebreak
$BA=\begin{pmatrix}\frac{1}{2} & \frac{1}{2}\\ -\frac{1}{2}i & \frac{1}{2}i\end{pmatrix}\begin{pmatrix}1 & i \\ 1 & -i\end{pmatrix}=\begin{pmatrix}(\frac{1}{2})(1)+(\frac{1}{2})(1) & (\frac{1}{2})(i)+(\frac{1}{2})(-i) \\ (-\frac{1}{2})(1)+(\frac{1}{2})(1) & (-\frac{1}{2})(i)+(\frac{1}{2})(-i) \end{pmatrix} = \begin{pmatrix}1 & 0 \\ 0 & 1\end{pmatrix} = I$\\
Since $B$ is a left and right inverse of $A$, $B=\begin{pmatrix}\frac{1}{2} & \frac{1}{2}\\ -\frac{1}{2}i & \frac{1}{2}i\end{pmatrix}$ is \textit{the} inverse of $A.$
\pagebreak
\section*{Problem 2}
$A = \begin{pmatrix}\frac{\sqrt{2}}{2} & -\frac{\sqrt{2}}{2}\\ \frac{\sqrt{2}}{2} & \frac{\sqrt{2}}{2}\end{pmatrix}$ and $\textbf{b} = \begin{pmatrix}0 \\ 2\sqrt{2}\end{pmatrix}$\\
For notation purposes, $\textbf{x}$ is a two dimensional vector s.t. $\textbf{x}=\begin{pmatrix}x \\ y\end{pmatrix}$.
\subsubsection*{a.}
Writing out $A\textbf{x}=\textbf{b}$ yields: $\begin{pmatrix}\frac{\sqrt{2}}{2} & -\frac{\sqrt{2}}{2}\\ \frac{\sqrt{2}}{2} & \frac{\sqrt{2}}{2}\end{pmatrix}\begin{pmatrix}x \\ y\end{pmatrix}=\begin{pmatrix}0 \\ 2\sqrt{2}\end{pmatrix}$\\
This yields two equations for each row of the $\textbf{x}$ vector:\\
$\frac{\sqrt{2}}{2}x -\frac{\sqrt{2}}{2}y=0$\\
$\frac{\sqrt{2}}{2}x +\frac{\sqrt{2}}{2}y=2\sqrt{2}$\\
\linebreak
From the first equation, we can determine that $x=y$.\\
Plugging into the second, we can ascertain that $\frac{\sqrt{2}}{2}x +\frac{\sqrt{2}}{2}x=2\sqrt{2}\implies\sqrt{2}x=2\sqrt{2}\implies x= 2 \implies y =2$\\
Therefore $\textbf{x}=\begin{pmatrix}2\\2\end{pmatrix}$
\subsubsection*{b.}
Since $\textbf{b}$ is the vector $\textbf{x}$ after having been rotated positive $45^{\circ}$, $\textbf{x}$ can be thought of as $\textbf{b}$ rotated negative $45^{\circ}$.\\
To calculate $\textbf{x},$ we can first calculate the "reversed" rotation matrix for $\theta = -45^{\circ} = -\frac{\pi}{4}$ as $\begin{pmatrix}\cos{-\frac{\pi}{4}} & -\sin{-\frac{\pi}{4}}\\ \sin{-\frac{\pi}{4}} & \cos{-\frac{\pi}{4}}\ \end{pmatrix} = \begin{pmatrix} \frac{\sqrt{2}}{2} & \frac{\sqrt{2}}{2}\\ -\frac{\sqrt{2}}{2} & \frac{\sqrt{2}}{2}
\end{pmatrix}$\\
Then calculate the "reversed" rotation matrix times $\textbf{b}$ to yield $\textbf{x}$:\\
$\begin{pmatrix} \frac{\sqrt{2}}{2} & \frac{\sqrt{2}}{2}\\ -\frac{\sqrt{2}}{2} & \frac{\sqrt{2}}{2}
\end{pmatrix}\begin{pmatrix}0 \\ 2\sqrt{2}\end{pmatrix}=  \begin{pmatrix} (\frac{\sqrt{2}}{2})(0) + (\frac{\sqrt{2}}{2})(2\sqrt{2})\\ -\frac{\sqrt{2}}{2})(0) + (\frac{\sqrt{2}}{2})(2\sqrt{2})
\end{pmatrix}=\textbf{x}=\begin{pmatrix}2\\2\end{pmatrix}$
\pagebreak
\subsubsection*{c.}
To prove that $A$ is an orthogonal matrix, we must prove that $AA^{T}=I=A^{T}A$.\\
The first necessary calculation is to find $A^{T}.$  $A^{T}=\begin{pmatrix}\frac{\sqrt{2}}{2} & \frac{\sqrt{2}}{2}\\ -\frac{\sqrt{2}}{2} & \frac{\sqrt{2}}{2}\end{pmatrix}$.\\
Then we calculate the LHS and RHS:\\
\linebreak
\textbf{LHS:}\\
$AA^{T}=\begin{pmatrix}\frac{\sqrt{2}}{2} & -\frac{\sqrt{2}}{2}\\ \frac{\sqrt{2}}{2} & \frac{\sqrt{2}}{2}\end{pmatrix}\begin{pmatrix}\frac{\sqrt{2}}{2} & \frac{\sqrt{2}}{2}\\ -\frac{\sqrt{2}}{2} & \frac{\sqrt{2}}{2}\end{pmatrix}=
\begin{pmatrix}
(\frac{\sqrt{2}}{2})(\frac{\sqrt{2}}{2})+(-\frac{\sqrt{2}}{2})(-\frac{\sqrt{2}}{2}) & 
(\frac{\sqrt{2}}{2})(\frac{\sqrt{2}}{2}) + (-\frac{\sqrt{2}}{2})(\frac{\sqrt{2}}{2}) \\
(\frac{\sqrt{2}}{2})(\frac{\sqrt{2}}{2})+(\frac{\sqrt{2}}{2})(-\frac{\sqrt{2}}{2}) & 
(\frac{\sqrt{2}}{2})(\frac{\sqrt{2}}{2}) + (\frac{\sqrt{2}}{2})(\frac{\sqrt{2}}{2})
\end{pmatrix} 
=\begin{pmatrix}\frac{1}{2}+\frac{1}{2} & \frac{1}{2}-\frac{1}{2} \\ \frac{1}{2}-\frac{1}{2} & \frac{1}{2}+\frac{1}{2}\end{pmatrix}=\begin{pmatrix}1 & 0 \\ 0 & 1\end{pmatrix}$\\
\linebreak
\textbf{RHS:}\\
$A^{T}A=\begin{pmatrix}\frac{\sqrt{2}}{2} & \frac{\sqrt{2}}{2}\\ -\frac{\sqrt{2}}{2} & \frac{\sqrt{2}}{2}\end{pmatrix}\begin{pmatrix}\frac{\sqrt{2}}{2} & -\frac{\sqrt{2}}{2}\\ \frac{\sqrt{2}}{2} & \frac{\sqrt{2}}{2}\end{pmatrix}=
\begin{pmatrix}
(\frac{\sqrt{2}}{2})(\frac{\sqrt{2}}{2})+(\frac{\sqrt{2}}{2})(\frac{\sqrt{2}}{2}) & 
(\frac{\sqrt{2}}{2})(-\frac{\sqrt{2}}{2}) + (\frac{\sqrt{2}}{2})(\frac{\sqrt{2}}{2}) \\
(-\frac{\sqrt{2}}{2})(\frac{\sqrt{2}}{2})+(\frac{\sqrt{2}}{2})(-\frac{\sqrt{2}}{2}) & 
(-\frac{\sqrt{2}}{2})(\frac{\sqrt{2}}{2}) + (\frac{\sqrt{2}}{2})(\frac{\sqrt{2}}{2})
\end{pmatrix} 
=\begin{pmatrix}\frac{1}{2}+\frac{1}{2} & -\frac{1}{2}+\frac{1}{2} \\ -\frac{1}{2}+\frac{1}{2} & \frac{1}{2}+\frac{1}{2}\end{pmatrix}=\begin{pmatrix}1 & 0 \\ 0 & 1\end{pmatrix}$\\
\linebreak
Therefore $A$ is an orthogonal matrix.\\  
And since it is orthogonal, $A^{-1}=A^{T}=\begin{pmatrix}\frac{\sqrt{2}}{2} & \frac{\sqrt{2}}{2}\\ -\frac{\sqrt{2}}{2} & \frac{\sqrt{2}}{2}\end{pmatrix}$.\\
\begin{equation*}
    \begin{split}
    A\textbf{x} &= \textbf{b}\\
    A^{-1}A\textbf{x} &= A^{-1}\textbf{b}\\
    I\textbf{x} &= A^{-1}\textbf{b}\\
    \textbf{x} &= A^{-1}\textbf{b}\\
    \textbf{x} &= \begin{pmatrix}\frac{\sqrt{2}}{2} & \frac{\sqrt{2}}{2}\\ -\frac{\sqrt{2}}{2} & \frac{\sqrt{2}}{2}\end{pmatrix}\begin{pmatrix}0 \\ 2\sqrt{2}\end{pmatrix}\\
    \textbf{x} &=\begin{pmatrix} (\frac{\sqrt{2}}{2})(0) + (\frac{\sqrt{2}}{2})(2\sqrt{2})\\ -\frac{\sqrt{2}}{2})(0) + (\frac{\sqrt{2}}{2})(2\sqrt{2})
\end{pmatrix}\\
    \textbf{x} &=\begin{pmatrix}2\\2\end{pmatrix}
    \end{split}
\end{equation*}
 
\end{document}