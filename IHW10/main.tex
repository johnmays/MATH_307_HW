\documentclass[fleqn]{article}
\usepackage[utf8]{inputenc}
\usepackage{natbib}
\usepackage{amsmath}
\usepackage{amssymb}

\title{\textbf{MATH 307: Individual Homework 10}}
\author{John Mays}
\date{03/10/21, Dr. Guo}

\begin{document}

\maketitle

\section*{Problem 1}
$v_1= \begin{bmatrix}-1\\1\\0\\2\end{bmatrix},$  $v_2= \begin{bmatrix}0\\1\\1\\1\end{bmatrix}$\\
$P_{v_1}(v_4)=\frac{\langle v_4, v_1 \rangle}{\langle v_1, v_1 \rangle}v_1=\frac{1+2}{1+1+4}v_1=\frac{1}{2}\begin{bmatrix}-1\\1\\0\\2\end{bmatrix}=\begin{bmatrix}-0.5\\ 0.5\\0\\1\end{bmatrix}$\\
$P_{v_4}(v_1)=\frac{\langle v_1, v_4 \rangle}{\langle v_4, v_4 \rangle}v_4=\frac{1+2}{1+1+1}v_1=v_1=\begin{bmatrix}0\\1\\1\\1\end{bmatrix}$

\section*{Problem 2}
$\{1,x,x^2\}$ is a linearly independent set, because no vector in the set can be expressed as a linear combination of the others.
\subsubsection*{Proof:}
$\alpha_1,\alpha_2,\alpha_3\in\mathbb{R}.$\\
$\alpha_1(1)+\alpha_2(x)+\alpha_3(x^2)=0 \implies \alpha_1=\alpha_2=\alpha_3=0$\\

\noindent Furthermore, the vectors are not mutually orthogonal.
\subsubsection*{Proof by counterexample:}
$\langle 1,x\rangle = \int_{0}^{1}1x=0.5$\\
$\langle v_j,v_k\rangle\neq 0 $ for $ j\neq k$, therefore the vectors cannot be orthogonal.
\pagebreak

\section*{Problem 3}
$v_1= \begin{bmatrix}0\\1\\1\end{bmatrix},$  $v_2= \begin{bmatrix}1\\0\\1\end{bmatrix},$  $v_3= \begin{bmatrix}1\\1\\0\end{bmatrix}$\\
$P_{v_3}v_1=\frac{\langle v_1,v_3 \rangle}{\langle v_3,v_3 \rangle}v_3=\frac{1}{2} \begin{bmatrix}1\\1\\0\end{bmatrix} = \begin{bmatrix}0.5\\0.5\\0\end{bmatrix}$\\
$P_{v_3}v_2=\frac{\langle v_2,v_3 \rangle}{\langle v_3,v_3 \rangle}v_3=\frac{1}{2} \begin{bmatrix}1\\1\\0\end{bmatrix} = \begin{bmatrix}0.5\\0.5\\0\end{bmatrix}$\\
Due to the linearity of the inner product,\\
$P_{v_3}(2v_1+v_2)=2P_{v_3}v_1+P_{v_3}v_2=2\begin{bmatrix}0.5\\0.5\\0\end{bmatrix}+\begin{bmatrix}0.5\\0.5\\0\end{bmatrix}=\begin{bmatrix}1.5\\1.5\\0\end{bmatrix}$
\section*{Problem 4}
\begin{equation*}
    e_1=\frac{1}{||1||}=1
\end{equation*}
Finding a vector orthogonal to $e_1$:
\begin{equation*}
\begin{aligned}
    v_2&=x-P_{e_1}x\\
    &=x-\frac{\langle x, 1\rangle}{\langle 1,1\rangle}1\\
    &=x-\frac{\int_0^1 x dx}{\int_0^1 1dx}1\\
    &=x-\frac{1}{2}
\end{aligned}
\end{equation*}
Finding the magnitude of $v_2$ in order to normalize it:
\begin{equation*}
\begin{split}
    \langle e_2,e_2 \rangle &= \int_0^1 (x-\frac{1}{2})^2dx\\
    &= \int_0^1 (x^2-x+\frac{1}{4})dx\\
    &= _0^1[\frac{1}{3}x^3-\frac{1}{2}x^2+\frac{1}{4}x]\\
    &=\frac{1}{3}-\frac{1}{2}+\frac{1}{4}=\frac{1}{12}\\
    \sqrt{\langle e_1,e_2 \rangle} &= \frac{1}{2\sqrt{3}}
\end{split}
\end{equation*}
Therefore $e_2=\frac{x-\frac{1}{2}}{\frac{1}{2\sqrt{3}}}=2\sqrt{3}x-\sqrt{3}$\\
Therefore the set $\{1, 2\sqrt{3}x-\sqrt{3}\}$ defines an orthonormal basis for V.


\end{document}