\documentclass[fleqn]{article}
\usepackage[utf8]{inputenc}
\usepackage{natbib}
\usepackage{amsmath}
\usepackage{amssymb}
\usepackage{relsize}

\title{\textbf{MATH 307: Group Homework 8}}
\author{Henry Zhong, Zhitao Chen, John Mays, Huaijin Xin}
\date{04/09/21, Dr. Guo}

\begin{document}

\maketitle

\section*{Problem 1}
\dots
\section*{Problem 2}
\dots
\section*{Problem 3}
$A\in F^{n \times n} = U\Sigma V^{*}$ where $U, V$ are unitary and $\Sigma$ is diagonal.\\
\begin{equation*}
    \begin{split}
        A^{*}A& =(U \Sigma V^{*})^{*}(U \Sigma V^{*})\\
        &=(V \Sigma^{*} U^{*})(U \Sigma V^{*})\\
        &= V \Sigma^{*} U^{*}U \Sigma V^{*}\\
        &= V \Sigma^{*} I \Sigma V^{*}\\
        &= V \Sigma^{*} \Sigma V^{*}\\ 
        &= V (\Sigma^{*} \Sigma) V^{*}
    \end{split}
\end{equation*}
Since $V$ is a unitary matrix, it is also invertible, and $V^{*}=V^{-1}$. Therefore,\\
\begin{equation*}
    V (\Sigma^{*} \Sigma) V^{*} = V (\Sigma^{*} \Sigma) V^{-1}
\end{equation*}
Furthermore, the resultant matrix of $\Sigma^{*}\Sigma$ will also be a diagonal matrix $=\text{diag}(|\sigma_1|^2,|\sigma_2|^2,\dots, |\sigma_n|^2)$\\
\linebreak
Since $V$ is invertible, $V^{*} = V^{-1}$, and $\Sigma^{*}\Sigma$ is a diagonal matrix, $V (\Sigma^{*}\Sigma) V^{*}$ is an eigendecomposition of $A^{*}A$.
\section*{Problem 4}
\dots

\end{document}