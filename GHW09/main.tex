\documentclass[fleqn]{article}
\usepackage[utf8]{inputenc}
\usepackage{natbib}
\usepackage{amsmath}
\usepackage{amssymb}
\usepackage{relsize}

\title{\textbf{MATH 307: Group Homework 9}}
\author{John Mays}
\date{04/16/21, Dr. Guo}

\begin{document}

\maketitle

\section*{Problem 1}
\dots
\section*{Problem 2}
\dots
\section*{Problem 3}
$A \in \mathbb{C}^{m \times n}$
\subsubsection*{SVD:}
    $A = U \Sigma V^{*}$ and $A^{*} = V \Sigma^{*} U^{*}$, where \\
    $\text{range}(A) = \text{span}(u_1, u_2, \dots, u_r) \subset \mathbb{C}^{m}$ and \\
    $\text{null}(A^{*}) = \text{span}(u_{r+1}, u_{r+2}, \dots, u_{m}) \subset \mathbb{C}^{m}$\\
\linebreak
Therefore,
\begin{equation*}
    \begin{split}
        \text{range}(A) + \text{null}(A^{*}) &= \text{span}(u_1, u_2, \dots, u_r) + \text{span}(u_{r+1}, u_{r+2}, \dots, u_{m})\\ 
        \text{range}(A) + \text{null}(A^{*}) &= \text{span}(u_{1}, u_{2}, \dots, u_{m})\\
        \text{range}(A) + \text{null}(A^{*}) &= \mathbb{C}^{m}\\
    \end{split}
\end{equation*}
And by the properties of SVD, we know that $U$ is an orthogonal matrix, therefore all the columns of $U$, $u_i$, are mutually orthogonal $\implies$ that the two collections of $u_i$, $\text{range}(A)$ and $\text{null}(A^{*})$, are orthogonal subspaces of $\mathbb{C}^{m}$.\\
\linebreak
Therefore $\text{range}(A) \oplus \text{null}(A^{*}) = \mathbb{C}^{m}$
%\begin{equation*}
%    \begin{split}
%
%    \end{split}
%\end{equation*}
\end{document}