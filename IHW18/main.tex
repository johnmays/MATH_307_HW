\documentclass[fleqn]{article}
\usepackage[utf8]{inputenc}
\usepackage{natbib}
\usepackage{amsmath}
\usepackage{amssymb}
\usepackage{relsize}

\title{\textbf{MATH 307: Individual Homework 18}}
\author{John Mays}
\date{04/14/21, Dr. Guo}

\begin{document}

\maketitle
%\begin{bmatrix}\end{bmatrix}

\section*{Problem 1}
\subsection*{Bases:}
\begin{equation*}
    \begin{split}
        A &= U\Sigma V^{*} \\
        A^{*} &= (U\Sigma V^{*})^{*} = V \Sigma^{*} U^{*} \\
    \end{split}
\end{equation*}
\begin{equation*}
    \begin{split}
        \text{range}(A) &= \{Ax | x \in \mathbb{F}^{n}\} = \text{span}(u_1,u_2, \dots, u_r)\\
        \text{range}(A^{*}) &=  \{A^{*}y | y \in \mathbb{F}^{m}\} = \text{span}(v_1,v_2, \dots, v_r)\\
    \end{split}
\end{equation*}
\subsection*{Proof that row rank = column rank:}
% Take $A^{m\times n}$, and say $A$ has column rank $r$, therefore a basis of $A$ could be described as $\{u_1,u_2,\dots, u_r\}$.  Take the matrix formed from these vectors as $\hat{U}$ of dimension $m \times r$.
Any $Ax = \mathlarger{\sum_{j=1}^{r}(\sigma_j v_j^{*} x)u_j} \in $ range$(A)$ according to its SVD.\\
and any $A^{*}y = \mathlarger{\sum_{i=1}^{r}(\sigma_i^{*} u_i^{*} y)v_i} \in $ range$(A^{*})$ according to its SVD.\\
\linebreak
Both sets of linearly independent vectors must have $r$ members.\\
Thus, for any $A$, the column space and row space both have dimension $r$, so their ranks are equal.
\pagebreak
\section*{Problem 2}
\subsection*{a)}
The column (and therefore row) rank of $A$ is five, because it has five nonzero singular values.\\
The set $\{v_1,v_2, \dots ,v_r\}$ is an orthonormal basis of range$(A^{*})$
\subsubsection*{Proof: }
\begin{equation*}
    \begin{split}
        A^{*} &= (U\Sigma V^{*})^{*} = V \Sigma^{*} U^{*}\\
    \end{split}
\end{equation*}
\subsection*{b)}
The nullity of $A^{*} = m - r = 1$ (by rank-nullity theorem).\\
Furthermore, a basis for null$(A^{*})$ = $\{u_{r+1}, u_{r+2}, \dots u_n\}$
\subsubsection*{Proof: }
\begin{equation*}
    \begin{split}
        A^{*} &= V \Sigma^{*} U^{*}\\
        A^{*}U &= V \Sigma^{*}\\
        A^{*}y &= \mathlarger{\sum_{i=1}^{r}(\sigma_{i} u^{*}_i y )v_i}\\
    \end{split}
\end{equation*}
And a vector $y$ can only be in the null space if $A^{*}y = \mathlarger{\sum_{i=1}^{r}(\sigma_{i} u^{*}_i y )v_i} = 0$, which would be only those values with $\sigma_i=0$, A.K.A: $\{u_{r+1}, u_{r+2}, \dots u_n\}$.

\end{document}