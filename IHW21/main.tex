\documentclass[fleqn]{article}
\usepackage[utf8]{inputenc}
\usepackage{natbib}
\usepackage{amsmath}
\usepackage{amssymb}
\usepackage{relsize}

\title{\textbf{MATH 307: Individual Homework 21}}
\author{John Mays}
\date{04/28/21, Dr. Guo}

\begin{document}

\maketitle

\section*{Problem 1}
$L_1=I+l_1e_1^{T}$\\
$L_1^{-1}=I-l_1 e_1^{T}$\\
If this is true, then $L_1 L_1^{-1} = L_1^{-1} L_1  = I$
\subsection*{Proof: }
\subsubsection*{First Part: $L_1 L_1^{-1}$}
\begin{equation*}
    \begin{split}
        L_1 L_1^{-1} &= (I+l_1e_1^{T})(I-l_1 e_1^{T})\\
        &= II + l_1e_1^{T}I-l_1e_1^{T}I - l_1 e_1^{T}l_1 e_1^{T}\\
        &= II - l_1 e_1^{T}l_1 e_1^{T}
    \end{split}
\end{equation*}
Multiplying $l_1 e_1^{T}$ by $l_1 e_1^{T}$ yields a 0 matrix, because every nonzero entry in the first row of $l_1 e_1^{T}$ is multiplied by a $0$ in the calculation of the resultant matrix:
\begin{equation*}
        l_1 e_1^{T}l_1 e_1^{T} = \begin{bmatrix}0 & \dots & 0\\ -l_{12} & \dots & 0 \\ \vdots & \ddots & 0\\ -l_{1n} & \dots & 0 \end{bmatrix}\begin{bmatrix}0 & \dots & 0\\ -l_{12} & \dots & 0 \\ \vdots & \ddots & 0\\ -l_{1n} & \dots & 0 \end{bmatrix} = \begin{bmatrix}0 & \dots & 0\\\vdots & \ddots & \vdots\\0 & \dots & 0\\\end{bmatrix}
\end{equation*}
Therefore, 
\begin{equation*}
    \begin{split}
        &= II - l_1 e_1^{T}l_1 e_1^{T}\\
        &= II - 0\\
        L_1 L_1^{-1} &= I
    \end{split}
\end{equation*}
\subsubsection*{Second Part: $L_1^{-1} L_1 $}
\begin{equation*}
    \begin{split}
        L_1^{-1} L_1 &= (I-l_1e_1^{T})(I+l_1 e_1^{T})\\
        &= II - l_1e_1^{T}I+l_1e_1^{T}I - l_1 e_1^{T}l_1 e_1^{T}\\
        &= II - l_1 e_1^{T}l_1 e_1^{T}
    \end{split}
\end{equation*}
From the first part of the proof, we know that $II - l_1 e_1^{T}l_1 e_1^{T} = II - 0 = I$
Therefore,
\begin{equation*}
        L_1^{-1} L_1 = I
\end{equation*}
\subsubsection*{Conclusion: }
$L_1 L_1^{-1} = L_1^{-1} L_1  = I$, therefore $L^{-1} = I-l_1 e_1^{T} $ is the inverse of 
$L_1=I+l_1e_1^{T}$.
\section*{Problem 2}
$A|b= \begin{pmatrix}-1&2&1&3&2\\-3&2&1&0&-5\\ -3&2&1&1&2\end{pmatrix}$\\
\linebreak
    \indent Subtract 3 times row 1 from row 2.\\
    \indent Subtract 3 times row 1 from row 3.\\
\linebreak
$= \begin{pmatrix}-1&2&1&3&2\\0&-4&-2&-9&-11\\ 0&-4&-2&-8&-4\end{pmatrix}$\\
\linebreak
    \indent Subtract row 2 from row 3.\\
\linebreak
$U| \widetilde{b}= \begin{pmatrix}-1&2&1&3&2\\0&-4&-2&-9&-11\\ 0&0&0&1&7\end{pmatrix}$\\
\subsubsection*{Finding a particular solution:}
$x_4 = 7$.  Say $x_3 = 0 \rightarrow -4x_2-2(0)-9(7)=-11 \rightarrow x_2 = -13$.  $-x_1+2(-13)+1(0)+3(7)=2 \rightarrow x_1 = 7$.\\
Therefore $x_p = \begin{pmatrix}7\\-13\\0\\7\end{pmatrix}$\\
To find a vector in the null space of A, I will solve
$\begin{pmatrix}-1&2&1&3&0\\0&-4&-2&-9&0\\ 0&0&0&1&0\end{pmatrix}$ for $x_3 = 1 \rightarrow x = \begin{pmatrix}0\\ -\frac{1}{2}\\1\\0\end{pmatrix}$\\
\linebreak
Therefore a general solution is $\begin{pmatrix}7\\-13\\0\\7\end{pmatrix} + \alpha \begin{pmatrix}0\\ -\frac{1}{2}\\1\\0\end{pmatrix}$
\section*{Problem 3}
\subsubsection*{a)}
$A|b= \begin{pmatrix}-1&2&1&0&2&-1\\2&0&0&3&-1&0\\-1&6&3&3&5&c\end{pmatrix}$\\
\linebreak
\indent Add 2 times row 1 to row 2\\
\indent Subtract row 1 from row 3\\
\linebreak
$= \begin{pmatrix}-1&2&1&0&2&-1\\0&4&2&3&3&-2\\0&4&2&3&3&c+1\end{pmatrix}$\\
\linebreak
$c$ must be $ = -3$.

\subsubsection*{b)}
Say $x_3=x_4=x_5=0$.  $\rightarrow 4x_2=-2 \rightarrow x_2 = -\frac{1}{2}$.  $\rightarrow -1x_1 +2(-\frac{1}{2})=-1 \rightarrow x_1 = 0$\\
Therefore, $x_p = \begin{pmatrix}0\\-\frac{1}{2}\\0\\0\\0\end{pmatrix}$

\end{document}