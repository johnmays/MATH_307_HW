\documentclass[fleqn]{article}
\usepackage[utf8]{inputenc}
\usepackage{natbib}
\usepackage{amsmath}
\usepackage{amssymb}
\usepackage{relsize}

\title{\textbf{MATH 307: Individual Homework 23}}
\author{John Mays}
\date{05/05/21, Dr. Guo}

\begin{document}

\maketitle

\section*{Problem 1}
\begin{equation*}
    \begin{split}
    A &= \begin{pmatrix}-2 & 1 \\ -1 & 2\end{pmatrix}, b= \begin{pmatrix}3\\3\end{pmatrix}\\
    x_1 &= \frac{\det(b|a_2)}{\det(A)} = \frac{(3)(2)-(1)(3)}{(-2)(2)-(1)(-1)} = \frac{3}{-3} = -1\\
    x_2 &= \frac{\det(a_1|b)}{\det(A)} = \frac{(-2)(3)-(3)(-1)}{(-2)(2)-(1)(-1)} = \frac{-3}{-3} = 1
    \end{split}
\end{equation*}
\textbf{Check: }
\begin{equation*}
    \begin{split}
    a_1x = -2x_1+x_2= -2(-1)+(1) = 3 \\
    a_2x = -x_1 + 2x_2 = -(-1)+2(1) = 3\\
    \end{split}
\end{equation*}
Therefore $x = \begin{pmatrix}-1 \\ 1 \end{pmatrix}$

\pagebreak
\section*{Problem 2}
\textbf{W.T.S:} $Ax = 0$ has nontrivial solutions $\implies 0$ is an eigenvalue of A.\\ 
\linebreak
If there exist an $x$ such that $Ax = 0$, then $\lambda x = 0 $ is true.  And since the corresponding $x$ is nonzero, $\lambda$ must be 0.\\
\linebreak
\textbf{W.T.S:} $0$ is an eigenvalue of A. $\implies Ax = 0$ has nontrivial solutions.\\ 
\linebreak
Assume $Ax = 0$ does not have nontrivial solutions.  If this is true, then $\lambda x = 0$ also has no solutions (if $\lambda$ could be zero, that would be a solution.\\
Therefore $Ax = 0$ does not have nontrivial solutions $\implies $ $\lambda \neq 0$.\\
Contrapositively, $\lambda = 0 \implies$ that $Ax = 0$ does have nontrivial solutions.\\
\linebreak
Therefore we can say that $Ax = 0$ has nontrivial solutions $\iff 0$ is an eigenvalue of A.\\
\linebreak
\textbf{W.T.S:} The determinant of $A = 0 \implies $ 0 is an eigenvalue of A.\\
$\det(A) = \lambda_1 \lambda_2 \dots \lambda_i \dots \lambda_n$.\\  
If $\det(A) = 0$, then one or more of the terms in the product must also be $0$.\\
Therefore $\det(A) = 0 \implies 0$ is an eigenvalue of $A$.\\
\linebreak
\textbf{W.T.S:} $0$ is an eigenvalue of $A \implies$ that the determinant of $A = 0$ \\ 
\linebreak
$\det(A) = \lambda_1 \lambda_2 \dots \lambda_i \dots \lambda_n$.\\
If any $\lambda_i = 0$, then $\det(A)$ must also be 0.\\
Therefore $0$ is an eigenvalue of $A \implies \det(A) = 0.$
\linebreak
Therefore we can now say The determinant of $A = 0 \iff 0$ is an eigenvalue of A.\\
Combining with our first iff. claim, we can now say that: The determinant of $A = 0 \iff 0$ is an eigenvalue of A $\iff Ax = 0$ has nontrivial solutions.
\section*{Problem 3}
From \textbf{Homework 22}, we have proved that full rank $\implies$ the matrix is\\ invertible.  Since $A$ is of full rank, $A$ is invertible.\\
Therefore if we take the normal equation:\\
\begin{equation*}
    A^{*}Ax =A^{*}b \rightarrow (A^{*}A)^{-1}(A^{*}A)x = (A^{*}A)^{-1}A^{*}b \rightarrow x = (A^{*}A)^{-1}A^{*}b
\end{equation*}
we can conclude that $x$ is unique for any given $A$ and $b$ because it is solely determined by $A$ and $b$.

\end{document}