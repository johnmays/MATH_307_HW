\documentclass[fleqn]{article}
\usepackage[utf8]{inputenc}
\usepackage{natbib}
\usepackage{amsmath}
\usepackage{amssymb}
\usepackage{relsize}

\title{\textbf{MATH 307: Individual Homework 16}}
\author{John Mays}
\date{04/05/21, Dr. Guo}

\begin{document}

\maketitle

% \begin{pmatrix}\end{pmatrix}

\section*{Problem 1}
$A=\begin{pmatrix}-2&-2\\-1&-3\end{pmatrix}$
\subsubsection*{Eigenvalues: }
\begin{equation*}
    \begin{split}
    &\det (A-\lambda I) = \det \begin{pmatrix}-2-\lambda&-2\\-1&-3-\lambda\end{pmatrix}= 0\\
    &(-2-\lambda)(-3-\lambda)-(-2)(-1) = 6+5\lambda+\lambda^2-2=0\\
    &\lambda^2+5\lambda+4=0\\
    &\lambda_1 = -1,\\
    &\lambda_2 = -4
    \end{split}
\end{equation*}
\subsubsection*{Eigenvectors: }
\begin{equation*}
    \begin{split}
    &Av_1=\lambda_1 v_1\\
    &\begin{pmatrix}-2&-2\\-1&-3\end{pmatrix}\begin{pmatrix}x\\y\end{pmatrix}=\lambda_1\begin{pmatrix}x\\y\end{pmatrix}\\
    &\begin{pmatrix}-2x-2y\\-1x-3y\end{pmatrix}=\begin{pmatrix}-x\\-y\end{pmatrix} \implies x= -2y\\
    &v_1=\begin{pmatrix}-2\\1\end{pmatrix}
    \end{split}
\end{equation*}
\begin{equation*}
    \begin{split}
    &Av_2=\lambda_2 v_2\\
    &\begin{pmatrix}-2&-2\\-1&-3\end{pmatrix}\begin{pmatrix}x\\y\end{pmatrix}=\lambda_2\begin{pmatrix}x\\y\end{pmatrix}\\
    &\begin{pmatrix}-2x-2y\\-1x-3y\end{pmatrix}=\begin{pmatrix}-4x\\-4y\end{pmatrix} \implies x=y\\
    &v_2=\begin{pmatrix}1\\1\end{pmatrix}
    \end{split}
\end{equation*}
\pagebreak

\section*{Problem 2}
\subsubsection*{Finding Eigenpair for $A^{-1}$:}
\begin{equation*}
    \begin{split}
        Av&=\lambda v\\
        A^{-1}A v &= \lambda A^{-1} v\\
        v &= \lambda A^{-1}v\\
        \frac{1}{\lambda}v&=A^{-1}v\\
        \lambda^{-1}v&=A^{-1}v \implies \text{the eigenpair for }A^{-1}\text{ is }( \lambda^{-1},v).
    \end{split}
\end{equation*}
\subsubsection*{Finding Eigenpair for $(A^{-1})^{3}$:}
(Proof by induction)\\
\linebreak
We know that $A^{-1}v =  \lambda^{-1}v$ is true.\\
Or, put another way, $(A^{-1})^{k}v =  (\lambda^{-1})^{k}v$ for $k=1$.\\
\linebreak
Assume this statement, $(A^{-1})^{k-1}v =  (\lambda^{-1})^{k-1}v$, is true for $k>1$.\\
\begin{equation*}
    \begin{split}
        (A^{-1})^{k}v &= (A^{-1})((A^{-1})^{k-1})v\\
        &= (A^{-1})((\lambda^{-1})^{k-1})v\\
        &= ((\lambda^{-1})^{k-1})(A^{-1})v\\
        &= ((\lambda^{-1})^{k-1})(\lambda^{-1})v\\
        (A^{-1})^{k}v &= (\lambda^{-1})^{k}v \text{ for all $k\geq 1$.}\\
    \end{split}
\end{equation*}
In our case, $k=3$, so a corresponding eigenvalue for $(A^{-1})^3$ is $(\lambda^{-1})^3=\lambda^{-3}$.\\
\linebreak
Thus, a corresponding eigenpair for $(A^{-1})^3$ is $(\lambda^{-3}, v)$.
\pagebreak
\section*{Problem 3}
We know that $Pv =  \lambda v, v\neq 0$ and $P^2=P$.
\subsubsection*{Proof (by induction):}
We know that $P^{k}v =  \lambda^{k} v$ for $k=1$.\\
\linebreak
Assume this statement, $P^{k-1}v =  P^{k-1}v$, is true for $k>1$.
\begin{equation*}
    \begin{split}
        P^{k}v &= PP^{k-1}v\\
        &= P\lambda^{k-1}v\\
        &= \lambda^{k-1}Pv\\
        &= \lambda^{k-1}\lambda v\\
        P^{k}v &= \lambda^{k}v \text{ for all $k\geq 1$.}\\
    \end{split}
\end{equation*}
Now we know $P^2v = \lambda^2 v$ (for $k=2$).\\
$Pv=P^2v=\lambda^2 v = \lambda v \implies \lambda^2=\lambda$, which only has two solutions: $0, 1$.\\
Therefore the eigenvalues for a projector may only be $0$ or $1$.

\section*{Problem 4}
\begin{equation*}
    \begin{split}
        A^{*}Av&=\lambda v\\
        v^{*}A^{*}Av&=v^{*}\lambda v\\
        A^{*}v^{*}Av&=\lambda v^{*} v\\
        (Av)^{*}(Av)&=\lambda v^{*} v\\
        \langle Av, Av \rangle&=\lambda \langle v, v \rangle\\
        (|| Av ||_2)^2&=\lambda (|| v ||_2)^2\\
        \lambda &= \frac{(|| Av ||_2)^2}{(|| v ||_2)^2}=(||A||_2)^2 \geq 0 \text{, by nonnegativity of matrix norms.}
    \end{split}
\end{equation*}
\end{document}